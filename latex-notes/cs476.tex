\documentclass[10pt]{article} 

\usepackage{fullpage}
\usepackage{bookmark}
\usepackage{amsmath}
\usepackage{amssymb}
\usepackage[dvipsnames]{xcolor}
\usepackage{hyperref} % for the URL
\usepackage[shortlabels]{enumitem}
\usepackage{mathtools}
\usepackage[most]{tcolorbox}
\usepackage[amsmath,standard,thmmarks]{ntheorem} 
\usepackage{physics}
\usepackage{pst-tree} % for the trees
\usepackage{verbatim} % for comments, for version control
\usepackage{tabu}
\usepackage{tikz}
\usepackage{float}

\lstnewenvironment{python}{
\lstset{frame=tb,
language=Python,
aboveskip=3mm,
belowskip=3mm,
showstringspaces=false,
columns=flexible,
basicstyle={\small\ttfamily},
numbers=none,
numberstyle=\tiny\color{Green},
keywordstyle=\color{Violet},
commentstyle=\color{Gray},
stringstyle=\color{Brown},
breaklines=true,
breakatwhitespace=true,
tabsize=2}
}
{}

\lstnewenvironment{cpp}{
\lstset{
backgroundcolor=\color{white!90!NavyBlue},   % choose the background color; you must add \usepackage{color} or \usepackage{xcolor}; should come as last argument
basicstyle={\scriptsize\ttfamily},        % the size of the fonts that are used for the code
breakatwhitespace=false,         % sets if automatic breaks should only happen at whitespace
breaklines=true,                 % sets automatic line breaking
captionpos=b,                    % sets the caption-position to bottom
commentstyle=\color{Gray},    % comment style
deletekeywords={...},            % if you want to delete keywords from the given language
escapeinside={\%*}{*)},          % if you want to add LaTeX within your code
extendedchars=true,              % lets you use non-ASCII characters; for 8-bits encodings only, does not work with UTF-8
% firstnumber=1000,                % start line enumeration with line 1000
frame=single,	                   % adds a frame around the code
keepspaces=true,                 % keeps spaces in text, useful for keeping indentation of code (possibly needs columns=flexible)
keywordstyle=\color{Cyan},       % keyword style
language=c++,                 % the language of the code
morekeywords={*,...},            % if you want to add more keywords to the set
% numbers=left,                    % where to put the line-numbers; possible values are (none, left, right)
% numbersep=5pt,                   % how far the line-numbers are from the code
% numberstyle=\tiny\color{Green}, % the style that is used for the line-numbers
rulecolor=\color{black},         % if not set, the frame-color may be changed on line-breaks within not-black text (e.g. comments (green here))
showspaces=false,                % show spaces everywhere adding particular underscores; it overrides 'showstringspaces'
showstringspaces=false,          % underline spaces within strings only
showtabs=false,                  % show tabs within strings adding particular underscores
stepnumber=2,                    % the step between two line-numbers. If it's 1, each line will be numbered
stringstyle=\color{GoldenRod},     % string literal style
tabsize=2,	                   % sets default tabsize to 2 spaces
title=\lstname}                   % show the filename of files included with \lstinputlisting; also try caption instead of title
}
{}

% floor, ceiling, set
\DeclarePairedDelimiter{\ceil}{\lceil}{\rceil}
\DeclarePairedDelimiter{\floor}{\lfloor}{\rfloor}
\DeclarePairedDelimiter{\set}{\lbrace}{\rbrace}
\DeclarePairedDelimiter{\iprod}{\langle}{\rangle}

\DeclareMathOperator{\Int}{int}
\DeclareMathOperator{\mean}{mean}

% commonly used sets
\newcommand{\R}{\mathbb{R}}
\newcommand{\N}{\mathbb{N}}
\newcommand{\Q}{\mathbb{Q}}
\newcommand{\su}[2]{\sum_{#1}^{#2}}
\renewcommand{\P}{\mathbb{P}}

\newcommand{\sset}{\subseteq}

\theoremstyle{break}
\theorembodyfont{\upshape}

\newtheorem{thm}{Theorem}[subsection]
\tcolorboxenvironment{thm}{
enhanced jigsaw,
colframe=Dandelion,
colback=White!90!Dandelion,
drop fuzzy shadow east,
rightrule=2mm,
sharp corners,
before skip=10pt,after skip=10pt
}

\newtheorem{cor}{Corollary}[thm]
\tcolorboxenvironment{cor}{
boxrule=0pt,
boxsep=0pt,
colback={White!90!RoyalPurple},
enhanced jigsaw,
borderline west={2pt}{0pt}{RoyalPurple},
sharp corners,
before skip=10pt,
after skip=10pt,
breakable
}

\newtheorem{lem}[thm]{Lemma}
\tcolorboxenvironment{lem}{
enhanced jigsaw,
colframe=Red,
colback={White!95!Red},
rightrule=2mm,
sharp corners,
before skip=10pt,after skip=10pt
}

\newtheorem{ex}[thm]{Example}
\tcolorboxenvironment{ex}{% from ntheorem
blanker,left=5mm,
sharp corners,
before skip=10pt,after skip=10pt,
borderline west={2pt}{0pt}{Gray}
}

\newtheorem*{pf}{Proof}
\tcolorboxenvironment{pf}{% from ntheorem
breakable,blanker,left=5mm,
sharp corners,
before skip=10pt,after skip=10pt,
borderline west={2pt}{0pt}{NavyBlue!80!white}
}

\newtheorem{defn}{Definition}[subsection]
\tcolorboxenvironment{defn}{
enhanced jigsaw,
colframe=Cerulean,
colback=White!90!Cerulean,
drop fuzzy shadow east,
rightrule=2mm,
sharp corners,
before skip=10pt,after skip=10pt
}

\newtheorem{prop}[thm]{Proposition}
\tcolorboxenvironment{prop}{
boxrule=0pt,
boxsep=0pt,
colback={White!90!Green},
enhanced jigsaw,
borderline west={2pt}{0pt}{Green},
sharp corners,
before skip=10pt,
after skip=10pt,
breakable
}

\setlength\parindent{0pt}
\setlength{\parskip}{2pt}


\begin{document}
\let\ref\Cref

\title{\bf{cs476}}
\date{\today}
\author{Austin Xia}

\maketitle
\newpage
\tableofcontents
\listoffigures
\listoftables
\newpage
\section{Course Information}
grade: 4 assignments
%lec2
\subsection{genral definition of financial deriustive contract}
A financial option/derivative is a financial contract at time t=0,
The value of the contract at future expiry T is determined by the market price of 
underlying asset at T


$S_t$ is underling price at time t, it is a stochoastic process.

Knowing the future value of contract in relation to the underlying price allows it to be used as an insurance

Holder bought insurance, how much should holder pay today

Writer bought risk (uncertainty) how much should the writer recieve

premium $V_0$

\subsection{payoff function}

$V_T=payoff(S_T)$

\begin{defn}[European option]
call option: right to buy at a preset price K at time T
put option: right to sell at a preset price L at time T
\end{defn}


\begin{defn}[American option]
The right can be exerciesed any time from now to exprity T
\end{defn}

\begin{defn}[holder]
    buyer of the option, enters a long position, +
\end{defn}

\begin{defn}[Writer]
    seller of the option, enters a short position, -

    eg. -100 share means sell 100 share of balabala
\end{defn}

let V(S(t),t) or $V_t$ denote the option value at time t.
recall S(t) is price of item at time t


question: is the first argument a function or a value

when know at T, $payoff(S_T)$

question: what is fair value $V_0$ of the option today 

What are payoff functions $$V_T=payoff(S_T)=max(S_T-K, 0)$$

there can be many payoff functions 


\subsection{one perioad binomial, fair value of option, arbitrage, put-call parity}

consider an 1-perioad binomial case 
Auume: T=1 and up probablity is p=0.1
20->22/18

consider a call with K=21

then option value today is 
$$.1*1+.9*0=.1$$

side 

cash account continously compounds at risk free rate 

borrowing moeny from a bank is selling a bond 

depositing money to bank == buying a bond 

let a bond has value $\beta(t)$ at time t,
$$\frac{d\beta(t)}{\beta(t)}=rdt$$

solving this ODE by integrating both sides,
$$log(\beta(T))-log(\beta(t))=r(T-t)$$

Discounting: 1 year ago, 
$$\beta(T)=1 \rightarrow \beta(t)=e^{-r(T-t)}$$

Discounting: 1 year ago, 
$$\beta(t)=1 \rightarrow \beta(T)=e^{r(T-t)}$$

An aaritage is trading opportunity to make a no-risk
profit greater than that of a bank deposit 
which earns interest $r \geq 0$ 

Example: buy one share of stock and borrow 100 (sell bonds)
$$H_0=1*S_0-100\text{ or } H_0=\{S_0, -100\}$$
the value at time t:
$$H_t=S_t-100e^{rt}$$

Mathematical characterization of an Arbitrage Strategy 

A profolio with initail $H_0=0$ but $H^T>0$ is arbitrage

question: interest rate?

\subsubsection{Put and Call Parity} 

Assume stock $S_t$ does not pay divident, interest rate $r\geq 0$,
no arbitrage. Then at any time $t\in(0,T)$ European call $C_t$ 
and put $P_t$, with same stike K and expiry T, on the same underlying,
satisfies 
$$C_t = P_t + S_t - Ke^{-r(T-t)}$$

$$C_t - P_t = S_T - K$$ 
and put them all back in time $T\rightarrow t$

\subsection{1-period in binomial model}
\begin{itemize}
    \item option replication and heging 
    \item computing option fair value 
    \item risk neutrual valuation
\end{itemize}

hedge the uncertainty

\subsubsection{Pricing by Replication}

qustion what is $V_t$
Assume:
\begin{itemize}
    \item $S_t>0$
    \item no arbitrage 
    \item length of time interval $\delta t>0$
\end{itemize}
stock: $S_t$ $S^u_{t+1}=uS_t$ $S^d_{t+1}=dS_t$ 

bond: $e^{-rt}\rightarrow 1$

option: $V_t$ $V^u_{t+1}=uV_t$ $V^d_{t+1}=dV_t$ 


At t, construct portfolio$\{\delta_tS_t, n_t\beta_t\}$ so that:

Buy $\eta$ bond and $\delta_t$ stock

$$\eta_t+uS_t\delta_t=V^n_{t+1}$$
$$\eta_t+dS_t\delta_t=V^d_{t+1}$$
n is bond, $\delta_t$ is amount of stock

Note solution of $\eta, \delta$ is unique 

No arbitrage, then 
$$V_t=\delta_tS_t+\eta_te^{-rt}$$
 in other words 
 $$\{V_t, -\delta_tS_t\}$$ is risk free

I can construct combination of stock and bond st 
the value of the stock+value of bond equals value of option at time t+1

recall $S_T^u = 22 =1.1S_0$
$S_T^d = 18 =0.9S_0$

r=0 K=21

we get 
$$22\delta+1\eta=1$$
$$18\delta+1\eta=0$$

$\delta=\frac{C^u_T-C^d_T}{(u-d)S_0}~~~\eta=-4.5$ (sell bond borrow cash)

\subsubsection{risk neutral valuation}
note: no arbitrage assumge implies $d\leq e^{rt}\leq u$

consider 
$$\psi^u+\psi^d=e^{-rt}$$
$$uS_t\psi^u+dS_t\psi^d=S_t$$

the unique solution is $\psi^u=e^{-rt}q^*$, $\psi^d=e^{-rt}(1-q^*)$
$$q*=\frac{e^{rt}-d}{u-d}~~q^*\in(0,1)$$

we can then treat q as a probability

we get 
$$S_t=e^{-rt}(q^*uS_t+(1-q^*)dS_t)=e^{-rt}E^Q(S_{t+1})$$

Where $E^Q$ uses $q^*$ as probablity


Let $\{\delta_tS_t, \eta_t\beta_t\}$ be replicating portfolio

by writing $\beta_t$ as $(q^*\centerdot 1+(1-q^*)\centerdot 1)$



$$V_t=E^Q(V_{t+1})$$

risk neutral valuation:
$$\beta_t=e^{-rt}\beta_{t+1}$$
$$S_t=e^{-rt}E^Q(S_{t+1})$$
$$V_t=e^{-rt}E^Q(V_{t+1})$$

where $V_t$ is derivative on S.

for the example 

risk neutrual probability is 
$$q^*=\frac{e^{rt}-d}{u-d}=\frac{1-0.9}{1.1-0.9}$$

\section{logarithmic return}
$$\Delta X_n=log(\frac{S_{n+1}}{S_n})~~~X_n=log(S_n) \Delta X_n=X_{n+1}-X_n$$

simple return $\frac{S_{n+1}}{S_n} \approx \Delta X_n$

so $\Delta X_n$ is $\Delta h$ with p and $-\Delta h$ with 1-p

\section{Discrete random walk and its limit}
price $S_0\rightarrow uS_0$

log price $X_0\rightarrow x_0+h$, both with probablity p

Properities of descrete random walk:
\begin{itemize}
    \item $\Delta X)n$ has identical independent distribution for any n

    \item moments:
    
    $E(\Delta X_n)=p\Delta h-(1-p)\Delta h=(2p-1)\Delta h=\alpha_0$

    $E(\Delta X^2_n)=p(\Delta h)^2+(1-p)(\Delta h)^2 = (\Delta h)^2$

    $var(\Delta X_n)=\Delta h^2-(2p-1)^2(\Delta h)^2$
\end{itemize}

$prob\left(\frac{\su{N-1}{n=0}\Delta X_n-N\alpha_0}{\sigma_0
\sqrt{N}}\leq x\right)\rightarrow F_{normal}(x)$

in finance, standard deviation is volatility $\sigma$, or variance $\sigma^2$

the expected log return in an interval of $\Delta t$ is $\alpha \Delta$, 
variance $\sigma^2 \Delta t$

so when $\sigma \alpha$ is given, we chan choose $p,\Delta h$ to satisfy $E(\Delta X_n)$ and $var(\Delta X_n)$

one frequent choice:
$$u=e^{\sigma\sqrt{\Delta t}}$$
$$d=e^{-\sigma\sqrt{\Delta t}}$$
$$p=1/2+\frac{1\alpha}{2\sigma}\sqrt{\Delta t}$$

note: \begin{itemize}
    \item option value does not depend on $\alpha$, it depends only on volatility 
    \item CRR leads to a nondrifting tree
    \item there are many differnet choices for (u,d,p) since there are 2 equation and 3 parameters 
    \item one can also set up no-arbitrage lattice(set p) for which the expected rate of return is r
 
\end{itemize}

an algorithm
\begin{itemize}
    \item choose u d p st. $log(u)=\Delta h=\sigma\sqrt{\Delta t}$
    $\Delta t$ small enough st $$u=e^{-\sigma\sqrt{\Delta t}}\leq e^{r\Delta t}\leq e^{\sigma\sqrt{\Delta t}}$$
    \item construct lattice of prices e.g. $u=e^{\sigma\Delta t}$
    $$S^{n+1}_{j+1}=... S^{n+1}_{j}=...$$
    \item at exprity T, value of option is $V^N_j=payoff(S_j^N)$
\end{itemize}

\section{Divident}
at t=0, the underlying pays divident D
$D=\rho S(t^-_d)$ at $t=t_d$, $0\leq t_d\leq T$

$$S(t_d^+)=S(t_d^-)-D$$

option is divident protected
$$V(S(t),t)\equiv V(t)\text{is continous funciton of t}$$ 
$$V(S(T_d^d),t_d^-)=V(S(T_d^+),t_d^+)=V(S(T_d^-)-D,t_d^+)$$

\subsection{brownian motion}
    $$dX=\alpha dt+\sigma dZ$$
    $\alpha dt$ is drift term, $\sigma$ is volatility $dZ$ is random term 
    $$dZ=\phi \sqrt{dt}~~\phi\sim N(0,1) $$

    $$E[dx]=\alpha dt$$
    $$Var[dx]=E[dX-E[dx]]^2=\sigma^2 dt$$

    discrete model: x has p possibility to go upward, q to go downward

    $$E[\Delta x]=(p-q)\Delta h$$
    $$E[\Delta X]^2 = \Delta h^2$$
    $$VAR[\Delta X]=4pq(\Delta h)^2$$
    $$E[X_n-X_0] = n(p-q)\Delta h=\frac{t}{\Delta t}(p-q)\Delta h$$
    $$VAR[X_n-X_0]=n4pq(\Delta h)^2$$

    if we take $\Delta h=\sigma \sqrt{\Delta t}$ and $p-q=\frac{\alpha}{\sigma}\sqrt{\Delta t}$
    $$p,q=0.5[1\pm \frac{\alpha}{\sigma}\sqrt{\Delta t}]$$
    $$E[X_n-X_0]=\alpha t$$
    $$Var[X_n-X_0]=t\sigma^2(1-\frac{\alpha^2}{\sigma^2}\Delta t)$$

\subsection{Standard Brownian motion}
    p=0.5 $\Delta X=\sqrt{\Delta T}$
    A standard Brownian motion $Z_t$ has 
    \begin{itemize}
        \item Z(0)=0
        \item $Z(t+\Delta t)-Z(t)\sim N(0,\Delta t)$
        \item $Z(t_2)-Z(t_1)$ is independent
    \end{itemize}
    Ito's Process 

    $dX_t=a\dotproduct dt+b\dotproduct dZ_t$
    a deterministic trend + random flucturation of standard Brownian

    Black Scholes Model:
    $$\frac{dS_t}{S_t}=\mu\dotproduct dt+\sigma\dotproduct dZ_t$$

    important propperties of $Z_t$, let $\psi_t\sim N(0,1)$
    $$dZ_t=\psi_t\sqrt{dt}~~~(dZ_t)^2=dt$$

    note variance of $\Delta Z_t^2$ goes to zero quadratically
\subsection{Geometric Brownian motion}
\subsection{Ito's lemma}

\section{}
\subsection{risk neutral pricing}
    $$E^Q(\frac{S(t_{n+1})}{S(t_n)})=e^{\Delta t}=1+r\Delta t+O((\Delta t)^2)$$
    this implies: under risk neutral probability,
    $$\frac{dS_t}{S_t}=rdt+\sigma dZ_t^Q$$
    $Z_t^Q$ is a standard Brwonain

    Under continous model, risk neutral pricing becomes 
    $$V(S,t)=e^{-r(T-t)}E^Q(payoff(S_T))$$
    where $E^Q()$ assumes underlying asset price follows  
    $$\frac{dS_t}{S_t}=rdt+\sigma dZ_t^Q$$
\subsection{MC for pricing European option}
    $$V(S_0,0)\approx \frac{1}{M}\su{M}{j=1}(e^{-rt}payoff((S_T)^j))$$

    recall we can generate $S_t=S_0e^{(r-0.5\sigma^2)t+\sigma Z_t}$

    we can generate M $S_T$ accordingly
\subsection{MC for pricing European path dependent option}
    \begin{defn}[Barrier Option]
        a barrier option come/ceases to exist when a barrier has been crossed e.g.(up/down, in/out)
    \end{defn}
    \begin{defn}[Up-Out]
        if $S_t$ crosses a up-barrier $S_u$, option pays nothing

        otherwise, standard payoff
    \end{defn}

    \begin{defn}
        Asian call payoff $$max(0,\frac{1}{T}\int^T_0S_tdt -K)$$
    \end{defn}

    we need to make this discrete 
    $$A_n=\frac{1}{n}((n-1)A_{n-1}+S(t_n))=\frac{n-1}{n}A_{n-1}+\frac{1}{n}S(t_n)$$

    If we use M simulations, sampling error is $O(\frac{1}{\sqrt{M}})$ becasue CLT
  
    CI:$$\hat\sigma=\left[ \frac{\su{M}{j=1}(Y^j-V^M)^2}{M-1} \right]^{\frac{1}{2}}$$

    approcimating,$$V^M-V^0\approx N(0,\frac{\hat\sigma}{\sqrt{M}})$$

    where $V_0=E(Y)$

    $$v_0\in \left[ V^M-1.96\hat\sigma/\sqrt{M},V^M+1.96\hat\sigma/\sqrt{M} \right]$$

    \section{}
    \subsection{Euler Method and Time Stepping}
        $$\frac{dS_t}{S_t}=rdt+\sigma(S_t,t)dZ_t$$
        integrate, we get (approximately)
        $$S_{n+1}=S_n+S_n(r\Delta t+\sigma(S_n, t_n)\sqrt{\Delta t}\phi_n)~~~\phi_n\sim N(0,1)$$
    \subsection{Error in Eular Method}
        the above equation comes from approximating integral from aboveabove equation
        $$\abs{E(S(T))-E(S_T^{\Delta t})}\leq C\Delta t$$
        S(T) is a price realization to 
        $$\frac{dS_t}{S_t}=rdt+\sigma(S_t,t)dZ_t$$

        $S_T^{\Delta t}$ is approximation with time step $\Delta t$
        $$4.8.2 \text{  course note}$$

        \emph{Balance Time Stepping Error and Sampling Error} 

        Assume a approximation has time stepping accuracy $\Delta t$
        Sample error is (M simulation) $O(\frac{1}{\sqrt{M}})$

        if we want total error = $O(\Delta t)$, we need $$M\approx \frac{C}{(\Delta t)^2}$$


        complexity = O(timestep * samples) 

        $=O((\frac{T}{\Delta t})M)$

        $=O(\frac{M}{\Delta t})$

        $=O(\frac{1}{(\Delta t)^3})$

        $$error=O(\frac{1}{complexity^{\frac{1}{3}}})$$

        thus to reduce error by factor of 10, we increase computation by $10^3$
    \subsection{Dynamic Trading Performance Analysis via MC simulation}

        in addition to back testing, we can perform hedging analysis based on
        a stochastic model

        Goal:\begin{itemize}
            \item How good is hedging strategy 
            \item how risky
            \item how we measure the risk
        \end{itemize}
\section{}
\subsection{dynamic hedging analysis}
    number of units in underlying is $$\delta^n_j=\frac{V^{n+1}_{j+1}-V_j^{n+1}}{(u-d)S_j}
    \approx \frac{dV}{dS}(S_j^n, t_n)$$

    initially, we have $$B_0=V_0-\delta_0S_0$$

    $$\text{protfolio }\pi = -V+\delta S+B\text{ has value }\pi_0=0$$

    $$\pi_{t_{n+1}^+} = \pi_{t_{n+1}^-}$$

    $$B_{n+1}=B_ne^{r\Delta t}+(\delta_n-\delta_{n+1})S_{n+1}$$
    interpretation: if $\delta_{n+1}>\delta_n$, buy $\delta_{n+1}-\delta_n$ units

    if $\delta_{n+1}<\delta_n$, sell $\delta_{n+1}-\delta_n$ units

    At T, liquid the portfolio which has value $$\pi_N=-V(S_N,t_N)+\delta_{N-1}S_N+B_{N-1}e^{r\Delta t}$$

    \begin{itemize}
        \item $\pi_N$ is random if it's 0, perfect hedge
        \item $\pi_N > 0$ , profit for writer
        \item $\pi_N > 0$ , a loss scenario 
        \item $\pi_n$ is hedging error, we consider discounted relative $P\&L$
         $$P\&L=\frac{e^{-rT}\pi_N}{V(S_0,0)}$$
    \end{itemize}
\subsection{computing delta under binomial lattice and MC pricing}
\subsection{Delta neutral, gamma neutral, bega neutral hedging}
    \subsubsection{Delta Neutral}
        $\pi=\{-V, \delta S, B\}$
        $$\frac{d\pi}{dS}=-\frac{dV}{dS}(S,t)-\delta\equiv 0$$

    \subsubsection{gamma neutral}
        $$\frac{d\pi}{dS}=\frac{d^2\pi}{dS^2}=0$$

        let $$\pi=\{-V, \delta_s S,\delta_i I, B\}$$
        $$-\frac{dV}{dS}+\delta_S+\delta_I\frac{dI}{dS}=0$$
        $$-\frac{d^2V}{dS^2}+0+\delta_I\frac{d^2I}{dS^2}=0$$
        hence $$di=\frac{\frac{d^2V}{dS^2}}{\frac{d^2I}{dS^2}}$$
    

\section{Models for correlated processes, generate correlated standard normal, simulate correlated process, BS PDE}

    Model for correlated process: 
    $$\frac{dS^1}{S^1}=\mu^1dt+\sigma^1dZ^1$$
    $$\frac{dS^2}{S^2}=\mu^1dt+\sigma^2dZ^2$$
    $$E(dZ^1dZ^2)=\rho dt$$

    to generate this model, we observe $dZ^i=\phi^i\sqrt{dt}$
    Thus we genrate 2 correlated standard normal with $$E(\phi^1 \phi^2)= \rho$$



    for any any $\sum \phi^ix_i$

    its mean is 0, its variance is $x^tQx$

    for positive semidefinte maxtrix Q, we can $Q=G^TG, G=chol(Q)$

    we generate $\phi=G^T\epsilon, \epsilon~_{iid}N(0,1)$

    we can show $\phi \phi^T = Q$

    \subsection{BD PDE}

    assume now volatility is a $\sigma(S_t, t)$

    if we construct profolio $\{V_t, -\delta_tS_t\}$ and choose $\delta_t=\frac{\delta V}{\delta S}$

    then we have $\delta \pi_t = f(S_t, t)$ does not have $\phi$ term (deterministic)

    and according to no arbitrage, $d\pi_t = r\pi_tdt$

    thus we have PDE, lec 13

 \end{document}
